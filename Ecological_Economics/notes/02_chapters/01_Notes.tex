\epigraph{"As economists we only see a part of the picture"}
{\textit{Monica Serrano Gutierrez}}

\section{Warming up}
\noindent
\begin{multicols}{2}
\subsection{Constanza et al (2015) Time to leave GDP behind}
by Costanza, R., I.Kubiszewski, E. Giovanini, H. Lovins, J. McGlade, K.E. Pickett, K.V. Ragnarsdóttir, D. Roberts, R. de Vogli \& R. Wilkinson (2014), Nature (\href{http://www.nature.com/news/development-time-to-leave-gdp-behind-1.14499}{link})
\epigraph{GDP measures "everything except that which makes life worthwhile"}
{\textit{Robert F. Kennedy}}
\noindent
GDP is a good measure for tne flow of everything that has a market price - mot as an indicator of well-being or environment.
\\
\textbf{Alternative measures} should take into account
\begin{itemize}
  \item Happiness
  \item Prosperity
  \item Environment
  \item Development
\end{itemize}
\subsection{Rodrik, D. (2015) Economics Rules: The Rights and Wrongs of the Dismal Science }
\noindent
An economist should have as many different models as possible in her toolbox\\
$\rightarrow$ choose the better model(s) for the specific research question.
\\
Our models are partial, thus, our conclusions are partial.

\subsection{The four laws of thermodynamis}
\begin{itemize}
  \item[\nth{1}] Law of thermodynamis: Energy can neither be created nor destroyed, but can change formms and flow from one place to another.
  \item[\nth{2}] Law of thermodynamis: The irreversibility of natural processes, and, in many cases, the tendency of natural processes to lead towards spatial homogeneity of matter and energy.
\end{itemize}
\noindent
Important works on environmental economics
\begin{itemize}
  \item Pigout (1920): Taxing externalities.
  \item Coase (NPE 1991): Contracting between parties.
  \item Elinor Ostrom (NPE 2012): Some communities use other mechanisms than the market for allocations etc. - and it's better than the market!
  \item Richard H. Thaler (NPE 2017): Behavioral economics (interests of firms).
  \item William Nordhaus (NPE 2018): For integrating climate change into long-run macroeconomic analysis.
\end{itemize}
\subsection{Environmental Economics vs. Ecological Economics}
\epigraph{"We cannot solve our problems with the same thinking we used when we created them"}{\textit{Albert Einstein}}
\noindent
Ecological Economics
\begin{itemize}
  \item Sustainability of the world as a whole.
  \item Looking at the world as a whole, i.e. no such thing as externalities.
\end{itemize}
Environmental Economics
\begin{itemize}
  \item Sustainability: Of the economy.
  \item Negative externalities: To the economy (the core).
  \begin{itemize}
    \item Uncompensated (adverse) impact of one person's action on the wellbeing of a bystander.
    \item Causes markets to be inefficient, and thus to maximize total surplus, e.g. pollution.
    \item Coase theorem: if private parties can bargain without cost over the allocation of resources, they can solve the problem of externalities on their own.
    \item Government action: Regulations (permits) or taxations (market correcting solution).
  \end{itemize}
\end{itemize}
\noindent
\textbf{The Climate:}\\
Average weather conditions that can be observed locally regionally or globally. Changes with or without human impact.
\\ \\
\textbf{Global warming:}
\begin{itemize}
  \item This is what is important!
  \item Designates the increase of average temperature
  \item \textbf{Global public good:}\\
  Standard solutions to tragedy of the commons:
  \begin{itemize}
    \item Price market-based policy: Carbon tax: Arthur Pigou (1920) \textit{The Economics of Welfare}
    \item Quantity market-based policy: Cap-and-trade system: Ronald Coase (1920) \textit{The problem of social cost}
    \item Alternative methods: Polycentric approach (consensus): Elinor Ostrom (2012) \textit{GLobal Environmental Commons} (NP, 2009).
  \end{itemize}
  Options to manage the "global common"
  \begin{itemize}
    \item Free rider problem: Westphalian nature of the current system of nations
    \item Problem of responsibiilty
  \end{itemize}
\end{itemize}
\textbf{History of international climate negotiations}\\
1987: Montreal: Agreement about the Freon gas - the only succesfull negotiation.
\end{multicols}


 % 2
\section{The economy as an open system}  % 2
Using Input-Output analysis to answer the questions.
\subsection{Growth, technology and the environment} % 2
\begin{multicols}{2}

\subsection{Economic growth and the environment} \noindent
Environmental Kuznets Curve
\begin{itemize}
  \item Classical Kuznets Curve: Inequality will rise with GDP growth - but will fall again.
  \item Evidence about the existence of an  is not conclusive (papers for and against).
  \begin{itemize}
    \item For: Looking at Freon and other CFCs, HCHCs and HFCs (Montreal, 1987).
    \item Against: Most work on GHGs.
    \item Increased international trade.
    \item Delocalization of $CO_2$: Moving industries $\rightarrow$ can increase emission intensity, but not true for all sectors!
  \end{itemize}
\end{itemize}
What has been the role of:
\begin{itemize}
  \item Technology?
  \item Population growth?
  \item Level of consumption per capita?
  \item Composition or structure of the consumption?
  \item Changes in trade structure?
\end{itemize}
\end{multicols}\noindent
\textcolor{red}{Insert graph of main determinants of change in global GHG in CO2-equivalents, s. 139!}\\
What about non-GHG?
\textcolor{red}{Insert: Drivers of emission growth for Spain 1995-2000, s. 143!}\\


\begin{multicols}{2}\noindent
\textbf{Innovation} - examples:
\begin{itemize}
  \item Energy (fuel) efficiency:
  \begin{itemize}
    \item Reduction of related emissions
    \item Rebound effect: Direct and indirect (sectors providing inputs to the sector)
    \item Jevons Paradox: Fuel is more effective $\rightarrow$ cheaper $\rightarrow$ more car-driving
  \end{itemize}
  \item Electric car:
  \begin{itemize}
    \item Reduction of local emissions
    \item Benefits for population health
    \item Provision of additional electric demand: coal vs. renewal? (a mix)
  \end{itemize}
  \item 3D printing
  \begin{itemize}
    \item Reduction of scrap or production waste.
    \item Reduction of emissions from transport.
    \item But: Do we end up with more consumption and end-of-life waste?
  \end{itemize}
\end{itemize}
\textit{You cannot think about the economy and the environment in linear terms!}
\end{multicols}
\subsection{Inequality, consumption \& environment}
\textcolor{red}{Demand-graph}
\begin{multicols}{2}
Does a more equal distribution increase pollution?
\begin{itemize}
  \item China: Middle-class increase consumption and pollution
  \item India: Religion plays a big role, e.g. vegetarian
  \item Engel Curve? The consumption changes from neccesities towards luxuries with income.
\end{itemize}
Policy maker: It is important to design
\begin{itemize}
  \item Climate policies does not increate economic inequalities
  \item Inequality reduction policies that do not increase GHG emissions.
\end{itemize}
Households' role is partially hidden in environmental statistics
\begin{itemize}
  \item Statistics based on territorial or production based-approach.
  \item Only \textit{direct} household emissions are considered (e.g. driving, cooking, painting).
\end{itemize}
\textcolor{red}{insert two graphs from Serrano (2008) \textit{Economic activity and atmospheric pollution in Spain: An Input–Output Approach}}\\

\textbf{Disaggregating the consumption vector}\\
\begin{itemize}
  \item Expenditure versus income?
  \begin{itemize}
    \item  How does savings/investments pollute?
  \end{itemize}
  \item Different size or compusition of households?
  \begin{itemize}
    \item Per-capita expenditure and emissions
    \item Equivalent expenditure and emissions
    \item Grouping households according to their size
    \item Multivariate regressions
  \end{itemize}
  \item Across countries
  \begin{itemize}
    \item E.g. consumption's share of health and education is not recorded if provided for free.
  \end{itemize}
  \item Bridge matrices?
  \begin{itemize}
    \item Different classifications, different criteria (micro data on households cannot be aggregated to the whole population)
  \end{itemize}
\end{itemize}
Difference between households
\begin{itemize}
  \item Income
  \item Expenditure
  \item Settlement, i.e. municipality size
  \begin{itemize}
    \item 2050: 2/3 of global population is expected to live in cities
  \end{itemize}
  \item Development related to growth etc.?
\end{itemize}
\end{multicols}\noindent
Spain 2000: The richer pollute more in absolute terms, but less in relative termns.
\textcolor{red}{insert graphs, s.169, 173, 180}
\begin{multicols}{2}\noindent
\subsection{Fragmentation, trade \& encvironment}\noindent
Emissions from \textbf{Transport}
\begin{itemize}
  \item International transport only grew a little
  \item Within-country transport exploded!
\end{itemize}
'Value added' measure can account for global value chains (intermediate goods).\\
\textbf{Conclusion} \\
\begin{itemize}
  \item It seems like global trade has little effect on emissions
  \begin{itemize}
    \item But you need to account for all of the product chain!
  \end{itemize}
\end{itemize}

\end{multicols}

% 3
\section{Price input-output model} % 3
\begin{multicols}{2}\noindent
Great because flows can be in all kinds of measures - we don't need to translate everything into Euroes.\\ \\ \\ \\ \\


\end{multicols}
% 4
\section{International Databases for the economy and the environment} % 4
\begin{multicols}{2}
...


\end{multicols}


% 4
\section{...} % 4
\begin{multicols}{2}


\end{multicols}


%\includegraphics[width = 1.0\textwidth]{CO2.PNG}
