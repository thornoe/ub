The standard definition of the ETI $\varepsilon$ is
\begin{align}
  \varepsilon=\frac{\delta z}{\delta(1-\tau)}\frac{(1-\tau)}{z}
  \label{eq:ETI}
\end{align}
Where $z$ is the taxable income and $\tau$ is the MTR.

The reported taxable income $z_{it}$ of individual $i$ at time $t$ can be expressed as a function (\ref{eq:income}) of potential income $z_{it}^0$ if the MTR $\tau_{it}$ was $0$ \citep{gruber2002elasticity} such that
\begin{align}
    z_{it} &= z_{it}^0\cdot(1-\tau_{it})^\varepsilon \nonumber \\
    \Rightarrow \log z_{it} &= \log z_{it}^0 + \varepsilon\cdot\log(1-\tau_{it}) \label{eq:income}
\end{align}
This log transformation is used for empirical estimation as it conveniently allows for the direct interpretation of $\varepsilon$ as the elasticity of taxable income with respect to the MTR, though in most cases ignoring the mostly theoretical potential income term.
