An extensive litterature has been conducted on the elasticity of labour supply with respect to the marginal tax rate (MTR) (see \cite{chetty2012bounds}) as labour supply is regarded a crucial component for growth in the long run and to avoid bottlenecks in the short run. Unfortunately, effects on the extensive margin and especially on the intensive margin is prone to misreporting and other measurement errors as well as frictions \citep{chetty2011adjustment}. The Elasticity of Taxable Income (ETI) not only is an attractive measure due to better availability and precision in register data but it also captures both the real responses of labour supply as well as the tax avoidance and tax evasion that affects tax revenue. While it can still be relevant to decompose the different effects, the ETI is a better measure of overall efficiency \citep{feldstein1999tax}. However, for efficiency analysis it is also necessary to take the effect on the total tax revenue into account as tax avoidance can lead to fiscal externalities that increase other tax bases due to income shifting towards capital or corporate income or simply inter-temporal substitution within the personal income base \citep{goolsbee2000happens,kreiner2014year}.

The natural experiments created by tax reforms serve as obvious subjects for investigation. Indeed, four of the five studies being reviewed in this survey pivots on three major tax reforms. Feldstein's (\citeyear{feldstein1995effect}) seminal paper and \citet{gruber2002elasticity} use panel data to examine the U.S. Tax Reform Act '86 (TRA86); the Danish 1987 Tax Reform \citep{kleven2014estimating}; as well as the Danish 2010 Tax Reform. \citep{kreiner2014year}.

We find that estimates of the ETI are vulnerable to identification problems related to endogeneity, mean reversion and non-tax-related changes in equality. Furthermore when controlling for most of theese effects Danish studies only find a small ETI and most, if not all, of it can be assigned to income-shifting rather than real responses.

In the following section the general theoretical model is layed out before the different empirical implementations are disentangled in section (\ref{sec:empirics}). The results of the different papers are presented and discussed in section (\ref{sec:results}) before the concluding remarks (\ref{sec:conclusion}).
