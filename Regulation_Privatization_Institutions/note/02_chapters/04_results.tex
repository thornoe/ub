\label{sec:results}
Though institutions and transparency appear very strong in Japan, there does not seem to be many obstacles that hinder politicians from also having business connections which, when all is said and done, is the very foundation for investigating whether there are signs of rent extraction from these connections. Even more so there is even less reason to expect that the family members of firms-owners would keep politics at arm's length. It will be up to the empirical analysis to judge If favoritism of family members' economic interests can be one reason that business people after all perceive the public sector in Japan to be mildly corrupt.

While \citet{faccio2006politically} did not find favourism to be significant the proposed empirical strategy of this paper is to take advantage of much richer data and variation over time in the political power of individual politicians. Furthermore the identification strategy besides from increasing robustness does also allow for various sample split results to investigate heterogeneity with respect to time and characteristics of municipalities, politicians, and firms.
