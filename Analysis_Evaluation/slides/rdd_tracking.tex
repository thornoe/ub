\documentclass[9pt]{beamer}

\usetheme[progressbar=frametitle]{metropolis}
\usepackage{appendixnumberbeamer}
\usepackage[style=authoryear, backend=bibtex8, natbib=true, maxcitenames=2]{biblatex}

\usepackage{graphicx}
\usepackage{import}

\usepackage{booktabs}
\usepackage[scale=2]{ccicons}

\usepackage[utf8]{inputenc}

\usepackage{pgfplots}
\usepgfplotslibrary{dateplot}

\usepackage{xspace}
\newcommand{\themename}{\textbf{\textsc{metropolis}}\xspace}

\usepackage{amsmath}
\usepackage{ulem} % use the "sout" tag to "strikethrough" text

\usepackage[super,negative]{nth} % allows writing 1st, 2nd, 3rd with superscript

  \makeatletter
  \renewcommand*{\@textcolor}[3]{%
    \protect\leavevmode
    \begingroup
      \color#1{#2}#3%
    \endgroup
  }
  \makeatother


\addbibresource{bib_rdd}

\numberwithin{equation}{section}

\title{Application of Regression Discontinuity Design}
\subtitle{The impact of tracking in Kenyan primary schools}
% \date{\today}
\date{\today}
\author{Thor Donsby Noe}
\institute{Analysis \& Evaluation of Public Policies}
%\institute{Department of Economics, University of Copenhagen}
% \titlegraphic{\hfill\includegraphics[height=1.5cm]{logo.pdf}}

    % \definecolor{BlueTOL}{HTML}{222255}
    \definecolor{BrownTOL}{HTML}{666633}
    \definecolor{GreenTOL}{HTML}{225522}
    % \setbeamercolor{normal text}{fg=BlueTOL,bg=white}
    \setbeamercolor{alerted text}{fg=BrownTOL}
    \setbeamercolor{example text}{fg=GreenTOL}

    \setbeamercolor{block title alerted}{use=alerted text,
        fg=alerted text.fg,
        bg=}
    \setbeamercolor{block body alerted}{use={block title alerted, alerted text},
        fg=alerted text.fg,
        bg=}
    \setbeamercolor{block title example}{use=example text,
        fg=example text.fg,
        bg=}
    \setbeamercolor{block body example}{use={block title example, example text},
        fg=example text.fg,
        bg=}

    \setbeamercolor{block title alerted}{use=alerted text,
        fg=alerted text.fg,
        bg=alerted text.bg!80!alerted text.fg}
    \setbeamercolor{block body alerted}{use={block title alerted, alerted text},
        fg=alerted text.fg,
        bg=block title alerted.bg!50!alerted text.bg}
    \setbeamercolor{block title example}{use=example text,
        fg=example text.fg,
        bg=example text.bg!80!example text.fg}
    \setbeamercolor{block body example}{use={block title example, example text},
        fg=example text.fg,
        bg=block title example.bg!50!example text.bg}


\begin{document}
\setbeamercolor{background canvas}{bg=white}
\maketitle


% ------------------------------------------------------------------------------
% ------ FRAME -----------------------------------------------------------------
% ------------------------------------------------------------------------------
\begin{frame}{Outline}
  \tableofcontents

\end{frame}


\section{Motivation}

\begin{frame}{Introduction}
  Duflo, E., P. Dupas, \& M. Kremer (2011) "Peer Effects, Teacher Incentives, and the Impact of Tracking: Evidence from a Randomized Evaluation in Kenya". In \textit{American Economic Review}, 101: 1739-1774.

  \textbf{'Tracking':} splitting up pupils according to prior achievements
  \begin{itemize}
    \item High-achieving pupils are widely regarded to gain from tracking
    \item Low-achieving pupils should be affected ambiguously
    \begin{itemize}
      \item[$\downarrow$] Less direct student-to-student spillovers \citep{epple2002ability}.
      % Which Epple, Newlon and Romano finds to dominate in the US.
      \item[$\uparrow$] Indirect effect: Teacher chooses an instruction level closer to pupil's ability \citep{figlio2002school, zimmer2003new, lefgren2004educational}.
      % Figlio and Lefgren find the two effects to cancel out, Zimmer finds the indirect effect to dominate.
    \end{itemize}
    \item Mid-achieving pupils are divided by the median $\rightarrow$ \textbf{discontinuity}
  \end{itemize}
\textbf{Randomized experiment} in Kenyan primary schools:
\begin{itemize}
  \item[$\rightarrow$] \citet{duflo2011peer} find that all quartiles receive a net benefit from tracking compared to the control group.
\end{itemize}
\end{frame}



\section{Theoretical model}

\begin{frame}{Model of educational outcome}
  \begin{itemize}
    \item[$y_{ij}:$] The educational outcome of a pupil $i$ in class $j$, given by
  \end{itemize}
  \begin{align}
    y_{ij}=x_{i}+f(\bar{x}_{-ij}) + g(e_{j})h(x_j^{*} - x_{i}) + u_{ij}
    \label{eq:outcome}
  \end{align}
  Where
  \begin{itemize}
    \item[$x_{i}:$] Prior test score of the pupil.
    \item[$\bar{x}_{-ij}:$] Average score of the other pupils in the class. $f(\bar{x}_{-ij}):$ is direct peer effect.
    \item[$e_{j}:$] Teacher's effort. $g(e_{j})$ is concave.
    \item[$x_j^{*}:$] The target level of teacher's instructions depending on class test scores.\\
      $h(\cdot):$ decreases to 0 when the difference between target and pupil's score is $x_j^{*} - x_{i}>\theta$.
    \item[$u_{ij}:$] i.i.d. stochastic pupil- and class-specific factors (symmetric, single-peak).
  \end{itemize}
\end{frame}

\begin{frame}{Teacher's utility maximization problem}
  The teacher decides on effort $e^{*}$ and target level $x^{*}$ to \textbf{maximize utility}.
  \begin{itemize}
    \item[$P(x^{*},e^{*}):$] Payoff function of the distribution of pupils' endline test scores.
    \item[$c(e^{*}):$] Cost function of effort (convex).
    \item[$\lambda>1:$] Contract teachers receive $\lambda$ times more payoff than civil service teachers.
  \end{itemize}
  The empirical results are \textbf{inconsistent} with three special cases:
  \begin{itemize}
    \item[-] No direct peer-effects.
    \item[-] No teacher response to class composition.
    \item[-] Teachers payoffs are linear or concave in students' test scores.
  \end{itemize}
  Results are \textbf{consistent} with a model where:
  \begin{itemize}
    \item[$\rightarrow$] Class composition has both direct and indirect effects.
    \item[$\rightarrow$] Teacher's payoffs are convex in student's test scores $\rightarrow$ target top of class.
  \end{itemize}
\end{frame}

\begin{frame}{Anticipated effects of tracking in general}
  The indirect effects depend on whether teachers are incentivized to target the top-, median- or low-achievers in a class (unaffected by treatment).
  \begin{itemize}
    \item High-achieving pupils should gain from tracking.
    \begin{itemize}
      \item[$\uparrow$] Direct student-to-student spillovers.
      \item[$\uparrow$] Indirect effect: Teacher increases effort and level.
    \end{itemize}
    \item Low-achieving pupils could be affected ambiguously
    \begin{itemize}
      \item[$\downarrow$] Less direct student-to-student spillovers.
      \item[$\uparrow$] Indirect effect: Teacher chooses instruction level closer to pupil's ability.
    \end{itemize}
    \item Mid-achieving pupils \textit{above} the median could be affected ambiguously
    \begin{itemize}
      \item[$\uparrow$] Direct student-to-student spillovers
      \item[$\uparrow \downarrow$] Indirect effect: Teacher might increase effort but also increase instruction level above pupil's ability. Depends on teacher's incentives (initial target).
    \end{itemize}
    \item Mid-achieving pupils \textit{below} the median could be affected ambiguously
    \begin{itemize}
      \item[$\downarrow$] Less direct student-to-student spillovers.
      \item[$\uparrow \downarrow$] Indirect effect: Teacher will lower the instruction level. Direction of effect depends on teacher's incentives.
    \end{itemize}
  \end{itemize}
\end{frame}

\begin{frame}{Effects of tracking in Kenya consistent with empirical results}
  \subsection{Effects consistent with empirical results}
  Incentive to maximize scores at the end of \nth{8} grade $\Rightarrow$ Kenyan teachers target the top-achievers in a class as many low- and medium-achievers drop out.
  \begin{itemize}
    \item High-achieving pupils gain from tracking
    \begin{itemize}
      \item[$\color{green} \uparrow$] Direct student-to-student spillovers.
      \item[$\color{lightgray} \uparrow$] Indirect effect: Teacher increases effort \sout{and level}.
    \end{itemize}
    \item Low-achieving pupils receive a net gain
    \begin{itemize}
      \item[$\color{lightgray} \downarrow$] Less direct student-to-student spillovers.
      \item[$\color{green} \uparrow$] Indirect effect: Teacher chooses instruction level closer to pupil's ability.
    \end{itemize}
    \item Mid-achieving pupils \textit{above} the median receive a net gain
    \begin{itemize}
      \item[$\color{green}\uparrow$] Direct student-to-student spillovers.
      \item[$\color{green}\uparrow \color{lightgray}\downarrow$] Indirect effect: Teacher might increase effort \sout{but also increase instruction level above pupil's ability}. \textit{Teachers initially target top-achievers anyway}.
    \end{itemize}
    \item Mid-achieving pupils \textit{below} the median receive a net gain
    \begin{itemize}
      \item[$\color{lightgray} \downarrow$] Less direct student-to-student spillovers.
      \item[$\color{green}\uparrow \color{lightgray} \downarrow$] Indirect effect: Teacher will lower the instruction level. \textit{Positive effect as teacher now targets mid-achievers as they are the top of the new class.}
    \end{itemize}
  \end{itemize}
\end{frame}



\section{Background}

\begin{frame}{Primary education in Kenya}
  \subsection{Kenyan school system}
  Characteristics
  \begin{itemize}
    \item Centralized education system
    \begin{itemize}
      \item National exams.
      \item Curriculum benefitting only high-achieving pupils \citep{glewwe2009many}.
    \end{itemize}
    \item Most teachers are hired centrally through the civil service
    \begin{itemize}
      \item Face weak incentives.
    \end{itemize}
    \item A minority of teachers are hired locally on short-term contracts.
    \begin{itemize}
      \item Face strong incentives $\rightarrow$ good track record can lead to a civil-service job.
    \end{itemize}
    \item Kenya recently abolished school fees $\rightarrow$ huge heterogeneity in pupils.
    \begin{itemize}
      \item Many \nth{1} generation learners.
      \item Few have attended preeschools (costly and optional).
    \end{itemize}
  \end{itemize}
  Incentives to target teaching to the top of the class
  \begin{itemize}
    \item Scores of own pupils in exit exam: A high rate drop out or repeat grades.
    \item Parents of top-achievers are more likely to interact with teachers.
    % as they're more similar
  \end{itemize}
\end{frame}

\begin{frame}{Experimental data}
  \subsection{Experimental data}
  \textbf{Experimental data:}
  \begin{itemize}
    \item In 2005 grants secured an extra teacher in 121 primary schools in Western Kenya with a single first-grade class that was split into two smaller classes.
    \item Random assignment into treatment:
    \begin{itemize}
      \item[T=1:] \textbf{Tracking:} Students were assigned to the two classes based on prior test scores, i.e. above median or below median (60 schools).
      \item[T=0:] \textbf{Control group:} Students were randomly assigned to the two classes (61 schools).
    \end{itemize}
  \end{itemize}
\end{frame}

\begin{frame}{Study design}
  \subsection{Study design}
\end{frame}


\begin{frame}{Estimation strategy}
  \subsection{Estimation strategy}
\end{frame}


\section{Results}


\begin{frame}{Main results}
\end{frame}



\begin{frame}{Additional evidence}
\end{frame}


\section{Concluding remarks}
\begin{frame}{Conclusion}
  \subsection{Conclusion}
\end{frame}


\begin{frame}{Policy implications}
  \subsection{Policy implications}
\end{frame}


\begin{frame}{Econometric takeaways}
  \subsection{Econometric takeaways}
Selection bias is not eliminated by controlling for intial test scores \citep{manning2006comprehensive}.
\begin{itemize}
  \item[$\rightarrow$] Need matching or experimental data with a low level of non-compliers.
\end{itemize}
\end{frame}


\section{References}

\begin{frame}%{References}
  \printbibliography
\end{frame}

%   \begin{figure}[!h]
%   %  \def\svgwidth{0.50\columnwidth}
%   %  \input{tree.pdf_tex}
%     \resizebox{3in}{!}{\input{tree.pdf_tex}}
%   %  \caption{Timeline illustration of event setup}
%   \end{figure}

\end{document}
