In Denmark there is strict regulation concerning all areas of public employment services. One such is that any job training program is not allowed to last for longer than 4 weeks in order to prevent companies from potentially taking advantage of unemployed persons as free labor while only postponing the actual job searching process. While there might be some validity to this concern in general, there could be potential benefits in prolonged job training programs for those who are long-term unemployed or at risk of becoming long-term unemployment.

Experiments in two Danish municipalities tested this hypothesis during the period 2014-2015. In section \ref{sec:background} I outline the experimental design and the theory behind is as well as the available data. In section \ref{sec:empirics} a Difference-in-Differences design is suggested and possible sources of bias and solutions to these are discussed before concluding in section \ref{sec:conclusion}.

%  o	Motivation of the project evaluation
