\label{sec:background}
\subsection{A "free municipality" experiment}
\label{subsec:experiements}
% o	Description of the policy
%  	Motivation and objectives of the policy
%  	Description of the program (objective and means)
%  	Rational behind the implementation of the policy in order to achieve objective goals.
As a part of round 1 of the Danish "free municipality" program two municipalities were exempted from the 4 weeks limit on the length of job training programs throughout 2014 and 2015. The motivation was to test the hypothesis that unemployed who are long-term unemployed or assessed to be in the risk of becoming long-term employed will get into lasting employment sooner through prolonged job training programs.
\\
\\
The means of the program was to give more agency back to, well, the agency. By reducing restrictions the public employment agencies gained increased flexibility to design individual job training programs substantiated through individual assessment.
\\
\\
There were a few differences in the design of the experiment between the two municipalities. The Municipality of Fredericia were able to offer up job training programs for up to 26 weeks but only in growth-companies with further potential of creating new jobs and growth.\footnote{\href{https://www.fredericia.dk/sites/default/files/16-676-27_fk_som_frikommune_2012-2015._rapport.pdf}{Fredericia as a Free Municipality 2012-2015: Summarizing report to the city council} (in Danish).} In the Municipality of Odense job training were offered for up to 13 weeks.\footnote{\href{https://www.odense.dk/politik/politikker-og-visioner/odense-som-frikommune}{Municipality of Odense: Evaluation of the Employment and Social Department's free municipality experiment} (in Danish).} Not only were public companies also included in the scope but they were the only ones that had to meet certain criteria besides from the individual employment agent's assessment of relevance and job potential.

In common for the two municipalities is that the intended target group are only those in long-term unemployed or those who are assessed as being at risk of long-term unemployment. Being long-term unemployed is a national definition according to not having performed any working hours in an ordinary job within the past 12 months. On the contrary, the latter criteria is entirely up to individual assessment, however, the outlined strategies for that assessment differed a bit in wording though they arguably might not in practice. In Frederica special attention was to be payed to those who were above 50 years old or those with a long university degree who had been unemployed for at least 13 weeks. In Odense the main reasonings were focused around recently having finished a qualifying education; those looking to change towards a different trade; or having no labor market experience.

\subsection{Anticipated effects of prolonged job training programs}
\label{subsec:theory}
% Discuss the theory behind the policy. Why the outcome variable should be affected by the program?
The theory that motivates the experiment is %from the individual's perspective
that loosening time restriction on job training programs [policy] would lead to individually designed job training programs that are prolonged only when individuals are likely to benefit from it [activity]. The prolonged programs should secure sufficient time to obtain qualifications and use them [initial outcome] letting the unemployed more credibly be able to document obtained qualifications in general and especially show the to the company in which they are in training [intermediate outcome] leading to sooner getting into ordinary employment or education [long-term outcome].

If the final effect is rejected each step of the theory should be investigated to find out where the assumption or implementation is off track.
% From the perspective of the company
%improve the obtainment of qualifications as well as .
%lead to better designed job training

% \subsection{Studies of job training}
% \label{subsec:studies}
% o	Brief summary of the literature (if exists)
%  	Methodology proposed for your evaluation
%  	Description of the methodology. Justification and rational of your choice.

\subsection{National and municipal level databases}
\label{subsec:data}
% o	Data and variables proposed
%  	Description of data sources.
%  	Description of data.
%  	Definition of variables
% Explain what type of data will be needed to evaluate the policy and whether you know this is available.
The experiment is only relevant for those who had been assured in order to receive the improved unemployment benefits named "dagpenge", a security scheme requiring a certain amount of performed working hours in an ordinary job within the past two years as well as paying membership to a an security unit, a so called "a-kasse". That is, prolonged job training programs were already available for those not eligible for "dagpenge" as this group is generally regarded to have a more loose connection to the job market.

Employment agencies in all municipalities has as a mandatory part of their procedure to assess the job readiness of unemployed. If placed in match-group 1 an individual is regarded job ready as opposed to match-group 2 as those ready to receive training or group 3 as in those who are temporarily passive. Only match-group 1 has been targeted by the experiment.
\\
\\
The official evaluation was carried out for each municipality separately using the municipal database following the individuals of the program to compare the average length of unemployment among the treated compliers (those accomplishing more than 4 weeks of job training) to the average length of unemployment in the total group of assured job ready unemployed in the municipality based on aggregated level data from \url{jobindsats.dk} where a series of refinements can be added for group selection. The discussion of evaluation design will in this paper revolve around this source of aggregated data. However, a more in-depth analysis could be performed using the weekly observations in the database for unemployment benefit receivers, "DREAM". Moreover, this can be merged with background variables from other administrative registries for micro econometric analysis.

When the experiments started in 2014 Odense was the \nth{4} most populace municipality in Denmark with 195,900 inhabitants and with 50,400 Fredericia was the \nth{36} most populace of all 98 Danish municipalities.\footnote{Statistics Denmark: Total inhabitants at \nth{1} of January.} For Odense the statistics used for the official evaluation is showed in table \ref{tab:unemployed}
\begin{table}[H]
  \centering
  \caption{Total and treated group in the Municipality of Odense}
  \footnotesize
    \begin{tabular}{lccc}
\toprule
Group / measure & 2013  & 2014  & 2015    \\
\midrule
Total: Assured job-ready unemployed & & & \\
\ Observations  & 14,359& 12,483& 12,447  \\
\ Share of full-time unemployed wrt. population
                & 3.7\% & 2.9\% & 2,8\%   \\
\ Average length of unemployment (weeks)
                & 17.1  & 16.7  & 15.8    \\
\midrule
Treated: Completing more than 4 weeks of job training & & & \\
\ Observations  & 0     & 5,455 & 4,861   \\
\ Average length of unemployment (weeks)
                & N/A   & 19.9  & 19.3    \\
\bottomrule
\end{tabular}
%
% \usepackage{float}
%
% \begin{table}[H]
%   \centering
%   \footnotesize
%     \begin{tabular}{lrrrr}
\toprule
Year, month   & Cities  & Towns & Villages  & Total \\
\midrule

\bottomrule
\end{tabular}
%
% \usepackage{float}
%
% \begin{table}[H]
%   \centering
%   \caption{Example of table}
%   \footnotesize
%     \begin{tabular}{lrrrr}
\toprule
Year, month   & Cities  & Towns & Villages  & Total \\
\midrule

\bottomrule
\end{tabular}
%
% \usepackage{float}
%
% \begin{table}[H]
%   \centering
%   \caption{Example of table}
%   \footnotesize
%     \input{04_tables/example}
%   \label{tab:ex}
% \end{table}

%   \label{tab:ex}
% \end{table}

%   \caption{Example of table}
%   \sourcec{}
%   \label{tab:example}
% \end{table}

  \sourcec{\url{Jobindsats.dk} and municipal database for Odense.}
  \label{tab:unemployed}
\end{table}
