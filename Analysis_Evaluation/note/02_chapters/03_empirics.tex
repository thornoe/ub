\label{sec:empirics}
% what method you think could be appropriate to evaluate that policy.
%  	Methodology proposed for your evaluation
%  	Description of the methodology. Justification and rational of your choice.
The official evaluation of the program was positive though the argument is only constructed by three elements: 1) only based on the results shown in table \ref{tab:unemployed} it is regarded a positive results that the average length of unemployment for those taking the prolonged job training program is only a little above the total group that also includes those at little to no risk of becoming long-term unemployed. 2) interviews with involved companies showed very positive feedback. 3) "the criteria for the scheme benefits those changing trade, newly unemployed and those assessed to be in risk of long-term unemployment".

The same evaluation was carried out by the same consultancy agency for both municipalities separately and no comparison to the trends in other municipalities were taken into account. Nonetheless the National Union of Municipalities as well the Ministry of Social and Internal Affairs adopted the conclusion and recommendation to continue the scheme in the municipalities and work towards implementing a similar policy through national reform as well.\footnote{\href{https://oim.dk/media/18130/slutevaluering-af-frikommuneordningen_lang-rapport.pdf}{Main report: Final evaluation of the free municipality scheme} (in Danish).} However, in the \nth{2} round of Free Municipality Experiments a different consultancy bureau is in charge of the monitoring and evaluation and there is improved focus on experiment design such that effect measurement is possible.\footnote{\href{https://www.kora.dk/aktuelt/undersoegelser-i-gang/projekt/i13208/Frikommuneenvaluering}{Evaluation of Free Municipality Experiments 2016-2020} (in Danish).}

\subsection{Aggregate level difference-in-differences}
The within-municipality comparison between treated and untreated job-ready unemployed in table \ref{tab:unemployed} does not surprisingly show that those who have attended a more than 4 weeks job training program does \textit{not} get into lasting employment sooner. This should be expected to be due to selection bias, as the employment agencies suggest the special program to those most likely to become or stay as long-term unemployed.

Instead we could regard the whole group of job-ready unemployed within each municipality as treated by having the conditional option of prolonged job-training. Thus, we can use difference-in-differences design to compare the development of employment among the treated as opposed to the remaining 96 untreated municipalities (not including the other municipality with a similar but slightly different experiment). This would allow us compare the effects of each of the two different implementations of the experiment. Also the two municipalities could be pooled together and compared to the remaining municipalities.

If tested and found that there is indeed not similar pre-treatment period trends in long-term unemployment for each of the two treated and the group of untreated municipalities a solution could be to include a series of municipality level controls. This would though imply actual difference-in-differences estimation using econometrics and not just comparison of different aggregates. As only two municipalities are in the control group the standard error of the treatment dummy is likely to be so big that the estimate is insignificant. An alternative is to use synthetic control methods to for each of the two treated municipalities build a synthetic counter-factual municipality with similar pre-level trends and characteristics from the remaining 96 municipalities.

\subsection{Micro level IV regression}
Using micro data the data access and cleaning would be a lot more time consuming without necessarily improving the validity. That is, as treatment is based on individual assessment, standard Regression in Discontinuity Design (RDD) cannot be used. Though, we could use IV-regression and fuzzy RDD to identify the likelihood of treatment based on background characteristics and thus compare how soon individuals get into lasting employment compared to individuals with similar characteristics in non-treated municipalities. Here we would not only be able to use municipal level controls but also rich individual controls.
