\documentclass[8pt]{beamer}

\usetheme[progressbar=frametitle]{metropolis}
\usepackage{appendixnumberbeamer}
\usepackage[style=authoryear, backend=bibtex8, natbib=true, maxcitenames=2]{biblatex}

\usepackage{graphicx}
\usepackage{import}

\usepackage{booktabs}
\usepackage[scale=2]{ccicons}

\usepackage[utf8]{inputenc}

\usepackage{pgfplots}
\usepgfplotslibrary{dateplot}

\usepackage{xspace}
\newcommand{\themename}{\textbf{\textsc{metropolis}}\xspace}

\usepackage{amsmath}
\usepackage{bm} % bold symbol in math mode

\usepackage{ulem} % use the "sout" tag to "strikethrough" text
\usepackage[super,negative]{nth} % allows writing 1st, 2nd, 3rd with superscript

\usepackage{tikz}
  \newcommand{\ditto}{
      \tikz{
          \draw [line width=0.12ex] (-0.2ex,0) -- +(0,0.8ex)
              (0.2ex,0) -- +(0,0.8ex);
          \draw [line width=0.08ex] (-0.6ex,0.4ex) -- +(-1.0em,0)
              (0.6ex,0.4ex) -- +(1.0em,0);
      }
  }

% Select what to do with command \comment:
   \newcommand{\comment}[1]{}  %comment not showed
  % \newcommand{\comment}[1]{\par {\bfseries \color{blue} #1 \par}} %comment showed

% Select what to do with todonotes: i.e. \todo{}, \todo[inline]{}
  % \usepackage[disable]{todonotes} % notes not showed
  \usepackage[draft]{todonotes}   % notes showed

  %\numberwithin{equation}{section}

\addbibresource{bib_tax}


\titlegraphic{\hfill\includegraphics[width= \textwidth]{logo}}
\title{Discrepancy in the Elasticity of Taxable Income}
\subtitle{A litterature survey}
\date{\today}
\author{Thor Donsby Noe}
\institute{Public Economics II: Theory of Taxation \\
          w. Javier Vázquez-Grenno}
%\institute{Department of Economics, University of Copenhagen}

    % \definecolor{BlueTOL}{HTML}{222255}
    \definecolor{BrownTOL}{HTML}{666633}
    \definecolor{GreenTOL}{HTML}{225522}
    % \setbeamercolor{normal text}{fg=BlueTOL,bg=white}
    \setbeamercolor{alerted text}{fg=BrownTOL}
    \setbeamercolor{example text}{fg=GreenTOL}

    \setbeamercolor{block title alerted}{use=alerted text,
        fg=alerted text.fg,
        bg=}
    \setbeamercolor{block body alerted}{use={block title alerted, alerted text},
        fg=alerted text.fg,
        bg=}
    \setbeamercolor{block title example}{use=example text,
        fg=example text.fg,
        bg=}
    \setbeamercolor{block body example}{use={block title example, example text},
        fg=example text.fg,
        bg=}

    \setbeamercolor{block title alerted}{use=alerted text,
        fg=alerted text.fg,
        bg=alerted text.bg!80!alerted text.fg}
    \setbeamercolor{block body alerted}{use={block title alerted, alerted text},
        fg=alerted text.fg,
        bg=block title alerted.bg!50!alerted text.bg}
    \setbeamercolor{block title example}{use=example text,
        fg=example text.fg,
        bg=example text.bg!80!example text.fg}
    \setbeamercolor{block body example}{use={block title example, example text},
        fg=example text.fg,
        bg=block title example.bg!50!example text.bg}


\begin{document}
\setbeamercolor{background canvas}{bg=white}
\maketitle


% ------------------------------------------------------------------------------
% ------ FRAME -----------------------------------------------------------------
% ------------------------------------------------------------------------------
\begin{frame}{Outline}
  \tableofcontents
\end{frame}


\section{Motivation}


\begin{frame}{Relevance}
  The elasticity of \textbf{labour supply} with respect to the Marginal Tax Rate (MTR)
  \begin{itemize}
    \item Important for growth in the long run and to avoid bottlenecks in the short run
    \item Prone to misreporting and frictions
  \end{itemize}
  The \textbf{Elasticity of Taxable Income (EIT)} wrt. the MTR
  \begin{itemize}
    \item Better availability and preciseness in administrative data
    \item Captures several behavioral responses
    \begin{itemize}
      \item Real responses of labour supply
      \item Tax avoidance, tax evasion
      \item Collective agreements and career choices
    \end{itemize}
    \item[$\rightarrow$] Good measure of overall efficiency
  \end{itemize}
  For efficiency analysis \textbf{total revenue} should be taken into account
  \begin{itemize}
    \item Tax avoidance can lead to fiscal externalities
    \begin{itemize}
      \item[$\rightarrow$] Increase other tax bases due to income shifting towards capital or corporate income
      \item[$\rightarrow$] Inter-temporal substitution within the personal income base
    \end{itemize}
  \end{itemize}
\end{frame}

\begin{frame}{The seminal paper}
  \begin{itemize}
    \item Natural experiment: The U.S. Tax Reform Act '86
    \begin{itemize}
      \item Reduced the marginal tax rate from 50 pct. to 28 pct. for high-earners
    \end{itemize}
    \item \citet{feldstein1995effect} estimate the elasticity of taxable income to be greater than one
    \begin{itemize}
      \item[$\rightarrow$] U.S. was on the wrong side of the Laffer curve prior to 1986
      \item[$\rightarrow$] Reducing the tax rate should have \textit{raised} the collected tax revenue
    \end{itemize}
    \item However, \citet{gruber2002elasticity} estimate ETI to 0.6 for high-earners in the 80s
  \end{itemize}
    \centering\includegraphics[width=0.8 \textwidth]{laffer}
\end{frame}

\begin{frame}{Discrepancy in estimates}
  Surveying 5 studies of the EIT, $\hat{\varepsilon}$
  \begin{itemize}
    \item A huge discrepancy is found in the estimates!
    \item Is a lower estimate due to better data availability and estimation methods?
    \item Or actual differences between tax reforms, institutional settings, and culture in the U.S. and Denmark respectively?
  \end{itemize}
    \begin{table}[h]
      \centering
      \footnotesize
        \begin{tabular}{lcc}
\toprule
{}                        & $\hat{\varepsilon}$ & Country \\
\midrule
Feldstein (1995)          & 1.04                & U.S.    \\
Gruber \& Saez (2002)     & 0.40-0.57           & U.S.    \\
Kleven \& Schultz (2014)  & 0.05-0.3            & DK      \\
Chetty et al (2011)       & 0.00                & DK      \\
Kreiner et al (2016)      & 0.00-0.08           & DK      \\
\bottomrule
\end{tabular}
% \begin{table}[h]
%   \centering
%   \footnotesize
%     \begin{tabular}{lcc}
\toprule
{}                        & $\hat{\varepsilon}$ & Country \\
\midrule
Feldstein (1995)          & 1.04                & U.S.    \\
Gruber \& Saez (2002)     & 0.40-0.57           & U.S.    \\
Kleven \& Schultz (2014)  & 0.05-0.3            & DK      \\
Chetty et al (2011)       & 0.00                & DK      \\
Kreiner et al (2016)      & 0.00-0.08           & DK      \\
\bottomrule
\end{tabular}
% \begin{table}[h]
%   \centering
%   \footnotesize
%     \begin{tabular}{lcc}
\toprule
{}                        & $\hat{\varepsilon}$ & Country \\
\midrule
Feldstein (1995)          & 1.04                & U.S.    \\
Gruber \& Saez (2002)     & 0.40-0.57           & U.S.    \\
Kleven \& Schultz (2014)  & 0.05-0.3            & DK      \\
Chetty et al (2011)       & 0.00                & DK      \\
Kreiner et al (2016)      & 0.00-0.08           & DK      \\
\bottomrule
\end{tabular}
% \begin{table}[h]
%   \centering
%   \footnotesize
%     \input{04_tables/tab_short}
%   \caption{Example of table}
%   \label{tab:elasticities}
% \end{table}

%   \caption{Example of table}
%   \label{tab:elasticities}
% \end{table}

%   \caption{Example of table}
%   \label{tab:elasticities}
% \end{table}

      % \caption{Example of table}
      % \label{tab:elasticities}
    \end{table}
\end{frame}


\section{Difference in differences}


\begin{frame}{\citet{feldstein1995effect}}
  The ETI is estimated as the treatment effect around the implementation of the TRA86
    \begin{align}
      \hat{\varepsilon}=\frac{\Delta \ln(z^H) - \Delta \ln(z^M)}{\Delta \ln(1-\tau^H) - \Delta \ln(1-\tau^M)}
      \label{eq:DD}
    \end{align}
  Evaluating the the relative differences from 1985 to 1988 of
  \begin{itemize}
    \item[$z^H:$] Income for high-earners
    \item[$z^M:$] Income for medium-earners.
    \item[$\tau^H$] The MTR for high-earners.
    \item[$\tau^M$] The MTR for medium-earners.
  \end{itemize}
  Panel of 3.538 medium-earners and 197 high-earners $\rightarrow$ robust?
\end{frame}

\begin{frame}{Issues with panel data}
  Panel are prone to bias from potential non-tax-related changes to income
  \begin{itemize}
    \item[1.] Mean reversion
    \begin{itemize}
      \item Individuals might only be in the high-income group initially as a results of an income shock - thus, would revert towards the mean
      \item[$\rightarrow$] Downward bias in the ETI estimate
    \end{itemize}
    \item[2.] Divergence in the income distribution
    \begin{itemize}
      \item Non-tax-related increases in inequality
      \item E.g. impacts of skill-biased technological change and globalization as observed in the U.S. in the 80s \citep{gruber2002elasticity}
      \item[$\rightarrow$] Upward bias in the ETI estimate
    \end{itemize}
  \end{itemize}
\end{frame}


\section{IV Panel Regresssions}


\begin{frame}{\citet{gruber2002elasticity}}
  Panel with $\sim$60,000 individuals $\rightarrow$ changes in the MTR throughout 1980s
  \begin{equation}
    \begin{split}
      \underbrace{\ln\left(\frac{z_{it+k}}{z_{it}}\right)}_\text{\%-change in income} &= \alpha_0 + \underbrace{\varepsilon\cdot \ln\left(\frac{1-\hat{\tau}_{it+k}}{1-\tau_{it}}\right)}_\text{ETI} + \underbrace{\sum_t\alpha_1 x_{it}}_\text{married} + \underbrace{\sum_t \alpha_2YEAR_t}_\text{year dummies} \\
       &+\underbrace{ \alpha_3\ln(z_{it}) + \sum_{d=1}^{10}\alpha_{4d}SPLINE_d(z_{it}) }_\text{controls for base year income} + u_{it},\ \ \ \ \ k=1,2,\text{ or }3
      \label{eq:IV}
    \end{split}
  \end{equation}
  Bias from \textbf{non-tax-related changes in inequality and mean reversion } is reduced by controlling for income in base year $t$
  \begin{itemize}
    \item[$\alpha_3$] log-income level
    \item[$\alpha_4$] 10 piece spline for decile of the income distribution
  \end{itemize}
  \textbf{Endogeneity problem}: An income shock $u_{it}>0$ $\rightarrow$ a mechanical rise in the MTR $\rightarrow$ corr$(\tau_{it},u_{it})>0$ $\rightarrow$ downward bias of $\varepsilon$
  \begin{itemize}
    \item[IV:] Use $\tau^h_{t+k}$ as an instrument for $\tau_{it+k}$
    \item[$\tau^h_{t+k}:$] MTR that individual $i$ would have paid in period $t+k$ due to changes in the tax system only $\rightarrow$ simulated using the NBER TAXSIM model.
  \end{itemize}
\end{frame}


\begin{frame}{\citet{kleven2014estimating}}
  \begin{itemize}
    \item Panels of all Danish taxpayers in all \textbf{three year periods} from 1984-2005.
    \item Merge individual level \textbf{administrative data} containing rich information about tax types, labour market, education, and sociodemographics.
    \item Danish setting: income \textbf{inequality more stable} over the 22-year period than even in other Nordic countries \\
    $\rightarrow$ Bias from non-tax related changes to inequality and mean reversion is likely to be much less than in prior studies
  \end{itemize}
\end{frame}


\begin{frame}{\citet{kleven2014estimating}}
  Baseline specification: differences at time $t$ are the differences between $t$ and $t+3$.
  \begin{equation}
    \underbrace{\Delta\ln z_{it}}_\text{\%-change in income} = \underbrace{\varepsilon\cdot\Delta\ln(1-\hat{\tau}_{it})}_\text{ETI} + \eta\cdot\Delta\ln y_{it} + \Delta\gamma_t^c \bm{x}_i^c + \gamma^v\cdot\Delta\bm{x}_{it}^v + \Delta u_{it}
    \label{eq:IV2}
  \end{equation}
  Similar to specification (\ref{eq:IV}) in \citep{gruber2002elasticity} with a few additions:
  \begin{itemize}
    \item[$\Delta \ln y_{it}:$] Difference in log virtual income (the sum of non-labour incomes)
    \begin{itemize}
      \item[$\rightarrow$] controls for \textbf{income shifting} towards capital or corporate income
    \end{itemize}
    \item[$\bm{x}_i^c:$] Time-invariant individual characteristics for which the effect $\gamma_t^c$ is allowed to change over time
    \item[$\Delta \bm{x}_it^v:$] Difference in time-variant individual characteristics for which the effect $\gamma^v$ is constant over time
  \end{itemize}
\end{frame}


\begin{frame}{\citet{chetty2011adjustment}}
  Look at all Danish wage earners during the smaller reforms of 1994-2001.

  Specification similar to equation (\ref{eq:IV2}) \citep{kleven2014estimating} with a few exceptions
  \begin{itemize}
    \item Merging employer and employee data $\rightarrow$ Adds occupation FE and region FE
    \item Fewer controls overall though
  \end{itemize}
  Main contribution:
  \begin{itemize}
    \item \textbf{Clear frictions} due to the Danish labour market being highly unionized
  \end{itemize}
\end{frame}


\section{Inter-temporal shifting}


\begin{frame}{\citet{kreiner2016tax}}
  The Danish 2010 Tax Reform
  \begin{itemize}
    \item The marginal tax rate was reduced by 7.5 pct. points
    \item Agreed upon as early as \nth{1} of March 2009 and passed in parliament by late May
    \item[$\rightarrow$] Self-employed were able to plan and employers to negotiate with their employees
  \end{itemize}
  Using Danish montly administrative data the short-run ETI is estimated using the panel regression: \begin{align}
    \underbrace{w_{y,m,i}}_\text{wage income} = \beta_0 + \underbrace{\varepsilon\cdot\frac{1-\tau_{y,i}}{1-\tau_{2009,i}}}_\text{ETI} + \underbrace{\beta_1\cdot d_{y,i}^{2010}}_\text{2010 dummy} + \underbrace{\beta_2\cdot d_i^T}_\text{treatment dummy} + u_{y,m,i}
    \label{eq:wage}
  \end{align}
  Evaluate the treatment group of 219,179 individuals against a control group of 109,500 individuals with weak or no incentives to shift their income due to the tax reform
  \begin{itemize}
    \item \textbf{Substantial shifting}: Estimate $\hat{\varepsilon}$ is 0.80 for D09-J10
    \item[$\rightarrow$] Omitting N09-J10 to estimate the short-run EIT without inter-temporal shifting
  \end{itemize}
\end{frame}


\section{Results}


\begin{frame}{Overview}
  \begin{table}
    \centering
    \footnotesize
    \begin{tabular}{lcccrcc}
\toprule
{}          & $\hat{\varepsilon}$ & Income group    & Method        & N         &  Period   & Country \\
\midrule
Feldstein (1995)          & 1.04  & $\sim\$100,000$ & DD            & 3,792     & 1985-1988 & U.S.    \\
Gruber \& Saez (2002)     & 0.40  & $>\$10,000$     & IV Panel Reg. &$\sim$60,000&1979-1990 & U.S.    \\
\ditto                    & 0.57  & $>\$100,000$    & \ditto        & \ditto    & \ditto    & \ditto  \\
Kleven \& Schultz (2014)  & 0.05  & wage earners    & IV Panel Reg. & 29,668,870& 1984-2005 & DK      \\
\ditto                    & 0.09  & self-employed   & \ditto        & 1,646,270 & \ditto    & \ditto   \\
\ditto                    & 0.11  & all taxpayers   & \ditto        & 11,799,628& 1984-1990 & \ditto  \\
\ditto                    &0.2-0.3& \ditto          & IV First-dif. & $\sim$3,000,000 & 1986-1989 & \ditto  \\
Chetty et al (2011)       & 0.00  & wage earners    & IV Panel Reg. & 8,302,905 & 1994-2001 & DK      \\
Kreiner et al (2016)      & 0.08  & highest quartile& Panel Reg.    & 328,679   & 2009-2010 & DK      \\
\ditto                    & 0.00  & \ditto          & \ditto              & \ditto    & 2009-2010*& \ditto  \\
\bottomrule
\end{tabular}
% \begin{table}[h]
%   \centering
%   \footnotesize
%     \input{04_tables/elasticities}
%   \caption{Example of table}
%   \label{tab:elasticities}
% \end{table}

    \caption{Estimated elasticity of taxable income in different studies. *excl. N09, D09 \& J10.}
    \label{tab:elasticities}
  \end{table}
  Most other reliable studies also find the ETI to be in the range 0.12-0.40 for the U.S. \citep{saez2012elasticity}
\end{frame}


\section{Conclusion}


\begin{frame}{Concluding remarks}
  \begin{itemize}
    \item Dfferences-in-differences estimation is a simple way to analyze effects in the proximity of a substantial tax reform
    \begin{itemize}
      \item But it can be difficult to completely exclude effects from non-tax-related changes to inequality and mean reversion
      \item Availability of controls as well as 2SLS panel regression over a period with a variety of tax system changes can reduce these biases.
    \end{itemize}
    \item U.S.: Most newer studies estimate a significant ETI of 0.12-0.40
    \item Denmark: Find modest effects $\rightarrow 0$ when omitting intertemporal income-shifting or self-employed
    \item Discrepancy can partly be explained
    \begin{itemize}
      \item By higher frictions and less options for tax avoidance in Denmark
      \item But estimates have also decreased with richer better data availability allowing for more controls.
    \end{itemize}
  \end{itemize}
  Implications for the U.S.
  \begin{itemize}
    \item[$\rightarrow$] Future studies might also find estimates closer to zero for the U.S., come richer data availability
    \item[$\rightarrow$] There might exist a revenue and efficiency loss from gaps in the U.S. tax law.
  \end{itemize}



\end{frame}


\section{References}
\begin{frame}%{References}
  \printbibliography
\end{frame}

%   \begin{figure}[!h]
%   %  \def\svgwidth{0.50\columnwidth}
%   %  \input{tree.pdf_tex}
%     \resizebox{3in}{!}{\input{tree.pdf_tex}}
%   %  \caption{Timeline illustration of event setup}
%   \end{figure}

\end{document}
