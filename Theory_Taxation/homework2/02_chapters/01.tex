The Danish tax system is ever-changing, however, minor changes does not seem sufficient to reveal the true long-run Elasticity of Taxable Income (EIT) due to transaction costs and other frictions \citep{chetty2011adjustment,kleven2014estimating}.

The mayor Danish tax reform within the past decades was without a doubt the 2010 Tax Reform where the marginal tax rate was reduced by 7.5 percentage points by January \nth{1} 2010. Due to the feature that the political agreement was concluded as early as May \nth{1} and passed in parlament by the end of May 2009 income-shifting effects and the short-run elasticity has already been thoroughly analyzed by Kreiner, Leth-Petersen, \& Skov (\citeyear{kreiner2014year,kreiner2016tax}). However there is still a potential to build on these studies in order to also estimate the long-run EIT by looking at several years following the reform.
